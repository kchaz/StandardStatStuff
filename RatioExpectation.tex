\documentclass{article}

\usepackage[margin = .75in]{geometry}



% Required for inserting images
\usepackage{graphicx} 

% Turn off indenting
\usepackage{parskip}

% Basic Math imports
\usepackage{amsthm}
\usepackage{amsmath}
\usepackage{amsfonts}
\usepackage{amssymb}
\usepackage{float}


%hyperlink stuff
%https://tex.stackexchange.com/questions/117943/hyperref-footnotes-and-section-ref-colors
\usepackage{etoolbox}
\usepackage{hyperref} %make hyperlinks colored, not boxed
%\hypersetup{colorlinks=true, urlcolor=blue}
\hypersetup{colorlinks=true,
             urlcolor=RoyalPurple,
             citecolor=RoyalPurple,
             linkcolor = RoyalPurple}
\usepackage[usenames,dvipsnames,svgnames,table]{xcolor}
\makeatletter
\def\@footnotecolor{red}
\define@key{Hyp}{footnotecolor}{%
 \HyColor@HyperrefColor{#1}\@footnotecolor%
}
\patchcmd{\@footnotemark}{\hyper@linkstart{link}}{\hyper@linkstart{footnote}}{}{}
\makeatother
\hypersetup{footnotecolor=gray}


%Some other imports
\usepackage{centernot} %for negation
\usepackage{arydshln} %dashed lines in tables
\usepackage{tikz} %drawing
\usepackage{multirow}
\usepackage{natbib} %bib


%Declare operators and additional notation
\DeclareMathOperator*{\argmax}{arg\,max}
\DeclareMathOperator*{\argmin}{arg\,min}

\newcommand{\indep}{\perp \!\!\! \perp}
\newcommand{\noindep}{\not\!\perp\!\!\!\perp}
\newcommand{\notdoubleimplies}{\centernot \longleftrightarrow}
\newcommand{\cOn}{\mathcal{O}_n}
\newcommand{\R}{\mathbb{R}}
\newcommand{\E}{\mathbb{E}}
\newcommand{\Var}{\mathbb{V}}
\newcommand{\cM}{\mathcal{M}}
\newcommand{\Cov}{\mbox{Cov}}
\newcommand{\Corr}{\mbox{Corr}}
\newcommand{\ITT}{\mbox{ITT}}
\newcommand{\simiid}{\stackrel{iid}{\sim}}
\newcommand{\greencheck}{{\color{ForestGreen}\checkmark}}
\newcommand{\redx}{{\color{BrickRed}$\times$}}

%environments
\newtheorem{theorem}{Theorem}
\theoremstyle{definition}
\newtheorem{definition}{Def.}

\title{Expectation of a Ratio}
\author{Kyla Chasalow}
\date{May 2025}

\begin{document}

\maketitle 

It is well known that 

\[ \E\left[\frac{X}{Y}\right]\neq \frac{\E[X]}{\E[Y]}\]

unless $\Var(Y)=0$. Now take a 2nd order Taylor expansion of $f(x,y)=\frac{x}{y}$ about $\mu_x=\E[X]$ and $\mu_y=\E[Y]$. The derivatives involved are:


\begin{align*}
    f^{(0)}(\mu_x,\mu_y) &= \frac{\mu_x}{\mu_y} \\
    f_x^{(1)}(\mu_x,\mu_y) &= \frac{1}{\mu_y}, \quad f_y^{(1)}(\mu_x,\mu_y) = -\frac{\mu_x}{\mu_y^2} \\
    f_{xx}^{(2)}(\mu_x,\mu_y) &= 0, \quad f_{xy}^{(2)}(\mu_x,\mu_y) = -\frac{1}{\mu_y^2}, \quad f_{yy}^{(2)}(\mu_x,\mu_y) = \frac{2\mu_x}{\mu_y^3}
\end{align*}


Thus, the second-order Taylor expansion is:

\begin{align*}
    \frac{X}{Y} \approx\ & \frac{\mu_x}{\mu_y} + \frac{1}{\mu_y}(X - \mu_x) - \frac{\mu_x}{\mu_y^2}(Y - \mu_y) \\
    & + 0 - \frac{1}{\mu_y^2}(X - \mu_x)(Y - \mu_y) + \frac{\mu_x}{\mu_y^3}(Y - \mu_y)^2
\end{align*}

Taking expectations of both sides and using $\E[X-\mu_x]=\E[Y-\mu_y]=0$,

\begin{align*}
    \E\left[\frac{X}{Y}\right] \approx\ & \frac{\mu_x}{\mu_y} - \frac{1}{\mu_y^2} \Cov(X, Y) + \frac{\mu_x}{\mu_y^3} \Var(Y)
\end{align*}

Equivalently, letting $\sigma_x^2=\Var(X),\sigma_y^2=\Var(Y)$,

\begin{align*}
    \E\left[\frac{X}{Y}\right] \approx\ & \frac{\mu_x}{\mu_y} - \frac{\sigma_x\sigma_y}{\mu_y^2} \Corr(X, Y) + \frac{\mu_x}{\mu_y^3} \sigma_y^2
\end{align*}


\newpage \textbf{Observations:}

\begin{itemize}


    \item In a first order Taylor sense, $\E\left[\frac{X}{Y}\right] \approx \frac{\mu_x}{\mu_y}$. 
    \item The second order reveals what could make this approximation better or worse. If $\mu_y^2 \gg \sigma_x\sigma_y\Corr(X,Y)$  and $\mu_y^3 \gg \mu_x\sigma_y^2$, then approximation will be better. This makes sense in that if $Y$ has small variance around a large mean, the ratio $X/Y$ will not be very sensitive to fluctuations in $Y$ and $Y$ will behave approximately like a constant.
    
    \item Even if $\Cov(X,Y)=0$, the expectation of a ratio is not the ratio of expectations.
    
    \item If it happens that

    \[ \Cov(X,Y)=\sigma_y^2\frac{\mu_x}{\mu_y}\]

    or equivalently, if
    
\[ \Corr(X,Y)=\frac{\sigma_y\mu_x}{\sigma_x\mu_y}\]
    

    Then the 2nd order Taylor approximation is exactly the ratio of means. 
    For example, this owuld happen if $X,Y$ are equal variance, their correlation has to happen to be the ratio of their means (which is not always possible if the ratio is $>1$ in magnitude). This edge case does not seem likely to often hold exactly.
    
    
\end{itemize}

\end{document}
